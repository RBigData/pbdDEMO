
\chapter{Phylogenetic Clustering}
\label{chp:phyclust}


{\it
``What one man can invent another can discover.''\\
\--- Sherlock Holmes
}


\section{Introduction}

Phylogenetic Clustering (Phyloclustering)\index{phyloclustering}
is a model-based clustering
technique combining evolutionary models to classify DNA/RNA sequences
~\ref{snoweye2011}.
Note that what speaking here, regarding to ``evolutionary'',
is a mathematical/statistical model to interpret biological targets.
Neither religion nor theology is involved. 

Phyloclustering assumes a sequence
$\bx_n = \{x_{n1}, x_{n2},\ldots, x_{nL}\}$
with $L$ loci
are observed from a population, but
may have $K$ subpopulations that similar sequence patterns are observed
within each common subpopulation.
Each subpopulation is represented by a common center sequence
$\bmu_k = \{\mu_{n1}, \mu_{n2},\ldots, \mu_{nL}\}$
with $L$ loci which may
or may not hypothetically exits in population and has to be determined.
Therefore, each sequence has a probability mutated/evolved from any
center sequence. The higher the probability, the closer to the center
sequence.

The evolutionary model is based on a continuous time Markov chain (CTMC)
for four nucleotides $\{\colorA, \colorG, \colorC, \colorT\}$
where the mutation process is characterized by
an instantaneously rate matrix $\bQ$ with dimension $4\times 4$.
The mutation chance from a nucleotide $x$ to a nucleotide $y$ in time $t$ is
$$
\Pr_{x, y}(t) = e^{\bQ_{x, y}t}.
$$
Assume each locus is mutated independently, then the mutation chance or
the transition probability from $\bmu_k$ to $\bx_n$ in time $t$ is
$$
p_{\bmu_k, \bx_n}(t) = \prod_{l = 1}^L \Pr_{\mu_{kl}, \bx_{nl}}(t).
$$
Suppose there are $K$ subpopulations with mixing proportion $\eta_k$'s, then
the mutation chance will be
$$
f(\bx_n; \bTheta) = \sum_{k = 1}^K \eta_k p_{\bmu_k, \bx_n}(t)
$$
where $\bTheta = \{\eta_1, \eta_2, \ldots, \eta_{K-1},
                   \bmu_1, \bmu_2, \ldots, \bmu_{K}, \bQ, t\}$.


Suppose observed $N$ sequences $\bx = \{\bx_1, \bx_2, \ldots, \bx_N\}$
selected from unknown $K$ subpopulations
with mixing proportion $\boldeta$ to be estimated,
then the log likelihood is
$$
\log L(\bTheta; \bx) = \sum_{k = 1}^K \log f(\bx_n; \bTheta)
$$
The unknown parameters are $\bmu_k$, $\bQ$, and $t$.

which describes the closeness of $\bx$ and $\bmu_k$.
The 

EM

model-based clustering

phylogenetics

evolution model




\section{Task Pull Parallelism}




\section{Bootstrap}

How much clusters?




\section{Exercises}
\label{sec:phyclust_exercise}

\begin{enumerate}[label=\thechapter-\arabic*]

\item


\end{enumerate}

