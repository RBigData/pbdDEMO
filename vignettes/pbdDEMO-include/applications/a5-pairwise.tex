\chapter{Pairwise Distance and Comparisons}
\label{chp:pairwise}

\inspire%
{An approximate answer to the right problem is worth a good deal more than
an exact answer to an approximate problem.}%
{John Tukey}


\section{Introduction}

Distance is not only a tool in geometry, but also appears in statistics. For
example, least square method in regression can be simply derived and computed
via Euclidean distance. The solution line is an approximate answer
in terms of minimum distance to all observations. Distance is also related
to a measure of two observations which describes relationship of both.
Usually, the smaller the closer. For example, the higher probability of
one virus evolving to a mutant means the smaller distance of two viruses
as described in Chapter~\ref{chp:phyloclustering}.
Further, distance method is simple to apply on clustering problems
and easy to visualize data structures such as K-means algorithm
introduced in Chapter~\ref{chp:pmclust}. For instance,
the observations of the same group are more similar in characteristics with
each other than those between different groups.


\section{Distributed Distance and Comparisons}

Suppose $x$ and $y$ are two observations and $d(x, y)$ is a distance or
a comparison of $x$ and $y$.
Although, it is efficient to compute a distance of any two observations
in \proglang{R} via \code{dist()} serially, it becomes non-trivial to
compute distance of distributed observations in parallel.

The potential problems are:
\begin{itemize}
\item
Communication must be evoked between processors when any two observations
are not located within the same processor.
\item
The resulting distance matrix may be too big
to held in one processor as data size increased even only a half (lower
triangular matrix is stored as row-major in a vector.)
\item
Compute all comparisons may be too time consuming even for small data sets. 
\end{itemize}

Distributed situations of observations and computed results (distance
matrix) are categorized next.
\begin{itemize}
\item Both observations and distance matrix are in one node and may both be
      in serial or in parallel within the node, typically SMP via OpenMP.
\item Observations are in common in all processors
      and distance matrix is distributed across nodes.
\item Observations are distributed across nodes
      and distance matrix is in common in all nodes.
\item Both observations and distance matrix are distributed
      across nodes.
\end{itemize}
Here, we may presume the distribution method is GBD row-major matrix (or
row-block major) as introduced in Section~\ref{sec:gbdstruct} since most of
native \proglang{R} functions can be extend and reused in such a way.

Note that the \code{dist()} only supports a few distance methods and assume
distance is symmetric by definition. However,
in practice, a more general measure may not be necessarily
symmetric of two observations. i.e. $d(x, y) \neq d(y, x)$.
In some cases, $d(x, x) \neq 0$ and the distance may also be dependent
on other measurements or conditions.


\section{Hierarchical Clustering}



\section{Phylogenetic Tree}



\section{Exercises}
\label{sec:pairwise_exercise}

\begin{enumerate}[label=\thechapter-\arabic*]

\item

\end{enumerate}


